\documentclass{article}

\usepackage{amsthm,amsfonts}
\usepackage{mathpartir}
\usepackage{mathtools}
\usepackage{tikz-cd}
\usepackage{hyperref,cleveref}
\usepackage{color,soul}
\usepackage{mathabx}

\theoremstyle{definition}
\newtheorem{definition}{Definition}[section]

% \theoremstyle{theorem}
\newtheorem{theorem}[definition]{Theorem}
\newtheorem{lemma}[definition]{Lemma}
\newtheorem{corollary}[definition]{Corollary}
\newtheorem{problem}[definition]{Problem}
\newtheorem{exercise}[definition]{Exercise}
\newtheorem{example}[definition]{Example}

\newenvironment{construction}{\begin{proof}[Construction]}{\end{proof}}



\newcommand{\types}{\mathcal T}
\newcommand{\type}{\ \textsc{type}}
\newcommand{\terms}{\mathcal S}
% \newcommand{\contexts}{\mathcal C}
% \newcommand{\context}{\textsc{ctx}}
\newcommand{\T}{\mathbb T}
\newcommand{\C}{\mathcal C}
\newcommand{\D}{\mathcal D}
\newcommand{\A}{\mathcal A}
\newcommand{\M}{\mathcal M}
\newcommand{\W}{\mathcal W}
\newcommand{\F}{\mathcal F}
\newcommand{\Set}{{\mathcal S}et}
\newcommand{\syncat}[1]{\C [#1]}
% \newcommand{\defeq}{\coloneqq}
% \newcommand{\interp}[1]{\lceil #1 \rceil}
\newcommand{\seq}{\doteq}
% \newcommand{\lists}{\mathcal Lists}
% \newcommand{\variables}{\mathcal Var}
\newcommand{\Epsilon}{\mathrm E}
\newcommand{\Zeta}{\mathrm Z}
\newcommand{\mor}{\mathrm {mor}}
\newcommand{\op}{\mathrm {op}}
\newcommand{\grpd}{\mathcal G}
\newcommand{\Id}{\mathtt {Id}}
\newcommand{\Path}{\mathrm {Path} \ }


\tikzset{pullbackcorner/.style={minimum size=1.2ex,path picture={
\draw[opacity=1,black,-,#1] (-0.5ex,-0.5ex) -- (0.5ex,-0.5ex) -- (0.5ex,0.5ex);%
}}}

\newcommand{\pullback}{\arrow[dr, phantom, "" {pullbackcorner} , very near start]}



\title{HW 4 (Mastermath HoTT)}
\author{Paige Randall North}



\begin{document}

\maketitle

\noindent\textbf{Problem 1.} Show the following.

\begin{lemma}
    \label{lem:equivalence relation}
    Nice path objects give reflexive and symmetric relations.

    If there are classes of morphisms $\mathcal L, \mathcal R$ such that (1) each $r_A \in \mathcal L$, (2) each $\epsilon_B \in \mathcal R$, (3) $\mathcal R$ is stable under pullback and closed under composition, (4) $\mathcal L$ is stable under pullback along $\mathcal R$, (5) each morphism to the terminal object is in $\mathcal R$, and (6) $\mathcal L \boxslash \mathcal R$, then we get an equivalence relation.
\end{lemma} 

\noindent\textbf{Problem 2.} Show the following.

\begin{lemma}
    \label{lem:factorization}
    Consider a weak factorization system $(\mathcal L, \mathcal R)$ on a category $\C$. We have that $\mathcal L$ is the class of all maps $f: X \to Y$ for which the following lifting problem has a solution.
    \[
         \begin{tikzcd}
             X \ar[r,"\lambda_f"] \ar[d,"f"] & Mf \ar[d,"\rho_f"]
             \\ 
             Y \ar[r,equal] & Y
         \end{tikzcd}
         \tag{$*$}
    \]

    Dually, $\mathcal R$ is the class of all maps $f: X \to Y$ for which the following lifting problem has a solution.
    \[
         \begin{tikzcd}
             X \ar[r,equal] \ar[d,"\lambda_f"] & X \ar[d,"f"]
             \\ 
             Mf \ar[r,"\rho_f"] & Y
         \end{tikzcd}
         \tag{$**$}
    \]
\end{lemma}

\noindent\textbf{Problem 3.} In about 250 words, explain both the advantages that cubical type theory holds over HoTT as well as the disadvantages.
\\

\noindent\textbf{Problem 4.} In about 250 words, explain why defining higher categories in HoTT has thus far not been successful, and give some examples of extensions of HoTT that have been proposed to ameliorate these problems.

\end{document}